\documentclass[twoside]{article}
\usepackage{amsmath,amssymb,amsthm,graphicx}
\usepackage{epsfig}
\usepackage[authoryear]{natbib}
\usepackage{enumitem}
\usepackage{mathabx}

\newcommand{\ind}[1]{1_{#1}} % Indicator function
\newcommand{\pr}{P} % Generic probability
\newcommand{\ex}{E} % Generic expectation
\newcommand{\var}{\textrm{Var}}
\newcommand{\cov}{\textrm{Cov}}
\newcommand{\sgn}{\textrm{sgn}}
\newcommand{\sign}{\textrm{sign}}
\newcommand{\kl}{\textrm{KL}} 
\newcommand{\abs}[1]{|{#1}|}

% Theorem-like declarations
\theoremstyle{plain}
\newtheorem{theorem}{Theorem}
\newtheorem{corollary}[theorem]{Corollary}
\newtheorem{lemma}[theorem]{Lemma}

\theoremstyle{definition}
\newtheorem{definition}[theorem]{Definition}
\newtheorem{example}[theorem]{Example}

\theoremstyle{remark}
\newtheorem{remark}[theorem]{Remark}

\renewcommand{\S}{\Sigma}
\renewcommand{\L}{\Lambda}
\renewcommand{\[}{\begin{equation}}
\renewcommand{\]}{\end{equation}}
\renewcommand{\b}{\backslash}
\newcommand{\g}{\,\vert\,}
\newcommand{\tr}{\mathrm{tr}}
\newcommand{\diag}{\mathrm{diag}}
\newcommand{\bea}{\begin{eqnarray}}
\newcommand{\eea}{\end{eqnarray}}
\newcommand{\hx}{\hat{x}}
\newcommand{\hxi}{\hat{\xi}}
\newcommand{\Var}{\mathrm{Var}}
\newcommand{\Cov}{\mathrm{Cov}}
\newcommand{\prop}{\propto}
\newcommand{\deq}{:=}

\newcommand{\EE}{\mathbb{E}}
\newcommand{\II}{\mathbb{I}}
\newcommand{\R}{\mathbb{R}}
\newcommand{\PP}{\mathbb{P}}

\newcommand{\La}{\mathcal{L}}

\newcommand{\n}{\mathcal{N}}

\newcommand{\bx}{\mathbf{x}}
\newcommand{\bX}{\mathbf{X}}
\newcommand{\by}{\mathbf{y}}
\newcommand{\bs}{\mathbf{s}}
\newcommand{\bn}{\mathbf{n}}
\newcommand{\br}{\mathbf{r}}
\newcommand{\bt}{\mathbf{t}}

\newcommand{\fig}[1]{Figure~\ref{fig:#1}}
\newcommand{\chap}[1]{Chapter~\ref{chap:#1}}
\newcommand{\mysec}[1]{Section~\ref{sec:#1}}
\newcommand{\app}[1]{Appendix~\ref{sec:#1}}
\newcommand{\eq}[1]{Eq.~(\ref{eq:#1})}
\newcommand{\eqs}[1]{Eqs.~(\ref{eq:#1})}
\newcommand{\eqss}[1]{(\ref{eq:#1})}
\newcommand{\thm}[1]{Theorem~\ref{thm:#1}}

\newcommand{\indep}{{\;\bot\!\!\!\!\!\!\bot\;}}
\newcommand{\eps}{\varepsilon}

\newcommand{\one}{1}
\newcommand{\Dir}{{\rm Dir}}
\newcommand{\Mult}{{\rm Mult}}
\newcommand{\Bin}{{\rm Bin}}
\newcommand{\Ga}{{\rm Ga}}
\newcommand{\IG}{{\rm IG}}
\newcommand{\InvGa}{{\rm IG}}
\newcommand{\Chisquare}{\Chi^2}
\newcommand{\St}{{\rm St}}
\newcommand{\Beta}{{\rm Beta}}
\newcommand{\iid}{i.i.d.}
\newcommand{\Eta}{{\cal N}}
\newcommand{\Ber}{{\rm Ber}}

\DeclareMathOperator*{\BP}{BP}
\DeclareMathOperator*{\DP}{DP}
\DeclareMathOperator*{\GP}{GP}
\DeclareMathOperator*{\BeP}{BeP}

% Caligraphic alphabet
\newcommand{\calr}{\mathcal{R}} % only because \cr already taken
\newcommand{\ca}{\mathcal{A}} \newcommand{\cb}{\mathcal{B}} \newcommand{\cc}{\mathcal{C}} \newcommand{\cd}{\mathcal{D}} \newcommand{\ce}{\mathcal{E}} \newcommand{\cf}{\mathcal{F}} \newcommand{\cg}{\mathcal{G}} \newcommand{\ch}{\mathcal{H}} \newcommand{\ci}{\mathcal{I}} \newcommand{\cj}{\mathcal{J}} \newcommand{\ck}{\mathcal{K}} \newcommand{\cl}{\mathcal{L}} \newcommand{\cm}{\mathcal{M}} \newcommand{\cn}{\mathcal{N}} \newcommand{\co}{\mathcal{O}} \newcommand{\cp}{\mathcal{P}} \newcommand{\cq}{\mathcal{Q}} \newcommand{\cs}{\mathcal{S}} \newcommand{\ct}{\mathcal{T}} \newcommand{\cu}{\mathcal{U}} \newcommand{\cv}{\mathcal{V}} \newcommand{\cw}{\mathcal{W}} \newcommand{\cx}{\mathcal{X}} \newcommand{\cy}{\mathcal{Y}} \newcommand{\cz}{\mathcal{Z}}

% Convergence
\newcommand{\convd}{\stackrel{d}{\longrightarrow}} % convergence in distribution/law/measure
\newcommand{\convp}{\stackrel{P}{\longrightarrow}} % convergence in probability
\newcommand{\convas}{\stackrel{\textrm{a.s.}}{\longrightarrow}} % convergence almost surely
\newcommand{\convr}{\stackrel{r}{\longrightarrow}} % convergence in r^{th} mean

\newcommand{\eqd}{\stackrel{d}{=}} % equal in distribution/law/measure
\newcommand{\argmax}{\mathop{\mathrm{argmax}}}
\newcommand{\argmin}{\mathop{\mathrm{argmin}}}
\newcommand{\conv}{\textrm{conv}} % for denoting the convex hull



\setlength{\oddsidemargin}{0.25 in}
\setlength{\evensidemargin}{-0.25 in}
\setlength{\topmargin}{-0.6 in}
\setlength{\textwidth}{6.5 in}
\setlength{\textheight}{8.5 in}
\setlength{\headsep}{0.75 in}
\setlength{\parindent}{0 in}
\setlength{\parskip}{0.1 in}

\newcommand{\lecture}[4]{
   \pagestyle{myheadings}
   \thispagestyle{plain}
   \newpage
   \setcounter{page}{1}
   \noindent
   \begin{center}
   \framebox{
      \vbox{\vspace{2mm}
    \hbox to 6.28in { {\Large \hfill #2  \hfill}  }
       \vspace{6mm}
       \hbox to 6.28in { {\Large \hfill #1  \hfill}  }
       \vspace{6mm}
       \hbox to 6.28in { {\it Author: #3 \hfill Date: #4} }
      \vspace{2mm}}
   }
   \end{center}
   \markboth{#1}{#1}
   \vspace*{4mm}
}

% Local Macros Put your favorite macros here that don't appear in
% stat-macros.tex.  We can eventually incorporate them into
% stat-macros.tex if they're of general use.

\begin{document}

\lecture{Learning an approximate posterior for NNs}{}{}{November 23, 2017}

\section{the GAN-way}

The motivation behind this is to learn a model for the approximate posterior $q(\mathcal{W})$ of a deep neural network. The learnt model can then be utilised for sampling different weights at test-time for robust predictions.

GANs are a way of approximating some unknown distribution, where you only need to be able to sample from the approximating distribution and take gradients of expectations through monte carlo estimates.

Let's say our target distribution is the predictive distribution from a Bayesin-NN, $p_{\text{exact}}(y|x)$. Now suppose we have a generator model which models this approximately. It takes as input a vector $x$ along with a set of noise-samples $\{\epsilon_i\}_{i=1}^M \sim p(\epsilon)$ and outputs $p_{\text{approx}}(y|x)$. This can be thought of as learning a prametric model $t_\theta$ for the approximate posterior,

\begin{align}
p_{\text{approx}}(y|x) &= \int p(y|x,W)q(W) dW\\
& = \int p(y|x,t_\theta(\epsilon))p(\epsilon)d\epsilon\\
&\approx \frac{1}{M} \sum_{m=1}^{M} p(y|x,t_\theta(\epsilon_m)), \ \epsilon \sim p(\epsilon)
\end{align}

(in short any expectation over $p_{\text{approx}}(y|x)$ can be reformulated as an expectation over $p(\epsilon)$).

Now suppose we have a discriminator $D_\psi$, to compare the two distributions, $p_{\text{approx}}$ and $p_{\text{exact}}$. We can thus write a min-max objective function as,

\begin{align}
\text{min}_\theta\ \text{max}_\psi\ \mathbb{E}_{y\sim p_{\text{exact}}(y|x)}[\log D_\psi(y)] + \mathbb{E}_{y\sim p_{\text{approx}}(y|x)}[\log \left( 1-D_\psi(y) \right) ]\\
\end{align}

The ``desired'' second term is the following, but are they equivalent? or can we sample moments of $W$ and then integrate under some gaussian assumptions?

\begin{align}
\mathbb{E}_{\epsilon\sim p_{\epsilon}}[\log \left( 1-D_\psi(f(t_\theta(\epsilon))) \right) ]\\
\end{align}

\section{Learning a parametric model MCMC transition distributions}


In a Markov Chain MC sampler, there is a ditribution $T(W^{t}|W^{t-1})$ which captures the transitions. The stationary distribution of this chain corresponds to the required distribution for which the sampler is developed which in our case is the approximate posterior $q(W)$. 

\begin{align}
q(W^t) = \int T(W^{t}|W^{t-1})q(W^{t-1}) dW
\end{align}

Is it possible to learn a NN-model for this transition distribution? Something similar to ``Neural Adaptive Sequential Monte Carlo [Gu et. al. 2014]''

% section distribution_perspective (end)

%\bibliographystyle{unsrt}
%\bibliography{citations}
% section siamese_networks (end)
\end{document}



