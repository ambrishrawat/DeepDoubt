\documentclass[twoside]{article}
\usepackage{amsmath,amssymb,amsthm,graphicx}
\usepackage{epsfig}
\usepackage[authoryear]{natbib}
\usepackage{enumitem}
\usepackage{mathabx}
\input{stat-macros}

\setlength{\oddsidemargin}{0.25 in}
\setlength{\evensidemargin}{-0.25 in}
\setlength{\topmargin}{-0.6 in}
\setlength{\textwidth}{6.5 in}
\setlength{\textheight}{8.5 in}
\setlength{\headsep}{0.75 in}
\setlength{\parindent}{0 in}
\setlength{\parskip}{0.1 in}

\newcommand{\lecture}[4]{
   \pagestyle{myheadings}
   \thispagestyle{plain}
   \newpage
   \setcounter{page}{1}
   \noindent
   \begin{center}
   \framebox{
      \vbox{\vspace{2mm}
    \hbox to 6.28in { {\Large \hfill #2  \hfill}  }
       \vspace{6mm}
       \hbox to 6.28in { {\Large \hfill #1  \hfill}  }
       \vspace{6mm}
       \hbox to 6.28in { {\it Author: #3 \hfill Date: #4} }
      \vspace{2mm}}
   }
   \end{center}
   \markboth{#1}{#1}
   \vspace*{4mm}
}

% Local Macros Put your favorite macros here that don't appear in
% stat-macros.tex.  We can eventually incorporate them into
% stat-macros.tex if they're of general use.

\begin{document}

\lecture{Learning a non-parametric approximate posterior for NNs}{}{}{November 23, 2017}

\section{the GAN-way}

The motivation behind this is to learn a model for the approximate posterior $q(\mathcal{W})$ of a deep neural network. The learnt model can then be utilised for sampling different weights at test-time for robust predictions.

GANs are a way of approximating some unknown target distribution, where you only need to be able to sample from the approximating distribution and take gradients of expectations through monte carlo estimates.

Let's say the target distribution is the predictive distribution from a Bayesin-NN, $p_{\text{exact}}(y|x)$. Our aim is to define a generative process that samples from $p_{\text{approx}}(y|x)$. Suppose we have a model $t_\theta$ which defines the approximate posterior, $q(W)$, i.e. if $\{\epsilon_i\}_{i=1}^M \sim p(\epsilon)$, then \{$W_i | W_i = t_\theta(\epsilon_i)\}_{i=1}^M\sim q(W)$. Subsequently, one could obtain a sample $y_i \sim p_{\text{approx}}(y|x)$ as $f(x,t_\theta(\epsilon_i))$. 

Now suppose we have a discriminator $D_\psi$, to compare the two distributions, $p_{\text{approx}}$ and $p_{\text{exact}}$. We can thus write a min-max objective function as,


\begin{align}
\text{min}_\theta\ \text{max}_\psi\ \mathbb{E}_{y\sim p_{\text{exact}}(y|x)}[\log D_\psi(y)] + \mathbb{E}_{y\sim p_{\text{approx}}(y|x)}[\log \left( 1-D_\psi(y) \right) ]\\
\end{align}

The ``desired'' second term needs to be reformulated in temrs of $p(\epsilon)$. In short any expectation over $p_{\text{approx}}(y|x)$ needs to be written as an expectation over $p(\epsilon)$), but are they equivalent? or can we sample moments of $W$ and then integrate under some gaussian assumptions?

\begin{align}
\mathbb{E}_{\epsilon\sim p_{\epsilon}}[\log \left( 1-D_\psi(f(t_\theta(\epsilon))) \right) ]\\
\end{align}


The problem is that while computing $y_i$ as $f(x,t_\theta(\epsilon_i))$, we are not marginalising over the weights. 
% It could take as input a vector $x$ along with a set of noise-samples $\{\epsilon_i\}_{i=1}^M \sim p(\epsilon)$ and outputs samples $y \sim p_{\text{approx}}(y|x)$. This can be thought of as learning a prametric model $t_\theta$ for the approximate posterior,

\begin{align}
p_{\text{approx}}(y|x) &= \int p(y|x,W)q(W) dW\\
& = \int p(y|x,t_\theta(\epsilon))p(\epsilon)d\epsilon\\
&\approx \frac{1}{M} \sum_{m=1}^{M} p(y|x,t_\theta(\epsilon_m)), \ \epsilon \sim p(\epsilon)
\end{align}

To make this more clear, for every $\epsilon_i$ what you get is a paired-sample $(y_i,W_i)$ from the joint distribution $p(y_i.W_i|x)$. Thus, marginalisation with respect to $W$ is a must here. Otherwise, your approximation model for posterior is a dirac on $W_i$.

\section{Learning a parametric model for transition distributions in MCMC}

In a Markov Chain MC sampler, there is a ditribution $T(W^{t}|W^{t-1})$ which captures the transitions. The stationary distribution of this chain corresponds to the required distribution for which the sampler is developed which in our case is the approximate posterior $q(W)$. 

\begin{align}
q(W^t) = \int T(W^{t}|W^{t-1})q(W^{t-1})dW
\end{align}

Is it possible to learn a NN-model for this transition distribution? Something similar to ``Neural Adaptive Sequential Monte Carlo [Gu et. al. 2014]''

\section{Learning a Variational Autoencoder for approximate posterior}

Suppose there is a parametric model which takes as input noise and outputs the mean and variance corresponding to the approximate posterior. (To be continued)

% section distribution_perspective (end)

%\bibliographystyle{unsrt}
%\bibliography{citations}
% section siamese_networks (end)
\end{document}



